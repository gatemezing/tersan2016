%%%%%%%%%%%%%%%%%%%%%%%%%%%%%%%%%%%%%%%%%%%%%%%%%%%%%%%%%%%%%%%%%%%%%%%%%%%%%%%%%%%%%%%%%%%%%%%%%%%%%%%%%%
%%%   %%%%
%%%%%%%%%%%%%%%%%%%%%%%%%%%%%%%%%%%%%%%%%%%%%%%%%%%%%%%%%%%%%%%%%%%%%%%%%%%%%%%%%%%%%%%%%%%%%%%%%%%%%%%%%%%

\documentclass[runningheads,a4paper]{llncs}

\usepackage[utf8]{inputenc}
\usepackage{amssymb}
\setcounter{tocdepth}{3}
\usepackage{graphicx}
\usepackage{tabularx}
\usepackage{url}
\usepackage{listings}
\usepackage{subfigure}
\usepackage{algorithmic}
\usepackage{algorithm}
\usepackage{verbatim}

%\newcommand{\keywords}[1]{\par\addvspace\baselineskip
%\noindent\keywordname\enspace\ignorespaces#1}

% todo macro
\usepackage{color}
\newtheorem{deflda}{Axiom}
\newcommand{\todo}[1]{\noindent\textcolor{red}{{\bf \{TODO}: #1{\bf \}}}}
\newcommand{\neon}{NeOn }
\newcommand{\protege}{Prot{\'e}g{\'e} }



%%%%%%%%%%%%%%%%%%%%%%%%%%%%%%%
%%%  Beginning of document  %%%
%%%%%%%%%%%%%%%%%%%%%%%%%%%%%%%

\begin{document}


\title{DECIPHER: A semantic transcoder for healthcare terminologies}

\author{Lydia Bascarane\inst{1}, Ghislain A. Atemezing\inst{1}, Florence Amardeilh\inst{1} }

\institute{
MONDECA, 35 Boulevard de Strasbourg, Paris, France. \\
%\and Fujitsu, Galway, Ireland.\\
\email{firsname.lastname@mondeca.com} \\
%\email{pierre-yves.vandenbussche@ie.fujitsu.com} \\
}

\maketitle


%%%%%%%%%%%%%%%%%%
%%%  Abstract  %%%
%%%%%%%%%%%%%%%%%%

\begin{abstract}

\todo{abstract}

\end{abstract}

\keywords{semantic interoperability, health information systems, reference terminology, dynamic mappings}


%%%%%%%%%%%%%%%%%%%%%%%%%
%%%  1. Introduction  %%%
%%%%%%%%%%%%%%%%%%%%%%%%%

\section{Introduction}
\label{sec:introduction}

\todo{INTRO HERE} \\

This paper presents a prototype for a vocabulary backed question answering system that can transform natural language questions into SPARQL queries, thus giving the end users access to the information stored in vocabulary repositories. The paper is structured as follows: Section \ref{sec:questions} describes the set of questions, followed by a system description in Section \ref{sec:system}. An evaluation is presented in Section \ref{sec:evaluation} and a short conclusion and future work in Section \ref{sec:conclusion}.  



%%%%%%%%%%%%%%%%%%%%%%%%%%%%%%%
%%%  2. Questions  %%%
%%%%%%%%%%%%%%%%%%%%%%%%%%%%%%%
\section{Related Work}
\label{sec:soa}


%%%%%%%%%%%%%%%%%%%%%%%%%%%%%%%
%%%  3. System Description  %%%
%%%%%%%%%%%%%%%%%%%%%%%%%%%%%%%

\section{System Description}
\label{sec:system}

\todo{add architecture}

\begin{figure}[ht!b]
\centering
\caption{Architecture of the system.}
%\includegraphics[scale=.6]{qa4lov-archi.pdf}
%\label{fig:q4lovarchi}
\end{figure}

The implementation uses the Quepy tool from Machinalis \cite{quepy2012}. The POS tagset used by Quepy is the Penn Tagset \cite{marcus1993building}. First, regular expressions are defined to match the natural language questions and transform them into an abstract semantic representation. Then, specific templates is defined to handle the questions that the system can handle. To handle regular expressions, Quepy uses the \texttt{refo} library\footnote{\url{https://github.com/machinalis/refo}} which work with regular expressions as objects. 

A vocabulary is defined by a fixed relation \texttt{voaf:Vocabulary}\footnote{All the prefixes are the ones used at \url{http://lov.okfn.org/dataset/lov/vocabs}} and a POS associated to a \texttt{vann:preferredNamespacePrefix}. LOV uses a unique prefix to identify namespaces, which is a string from length 2 to length 17, although the recommendation for publishers is to use a prefix with less than 10 characters \cite{pybernard12}.  

The syntactic processor is based on regular expressions using POS terms. As a vocabulary is identified by its prefix, we use  the following syntactic patterns: NN, NNS, FW, DT, JJ and VBN. Each question Q1 to Q14 is associated to a unique template. After, when a prefix is recognized, the semantic interpreter uses fixed relations with the English tag, which are properties in RDF triple pattern. Table \ref{tab:propTable} presents the different fixed relations currently used in the system to cover the set of the 14 questions.

\begin{table}
\centering
\caption{Relationship between Questions and Properties used to generate the SPARQL query from natural language}
\label{tab:propTable}
\begin{tabular}{|c|l|} \hline
\textbf{Question ID} & \textbf{Property} \\ \hline
\texttt{Q1} & \texttt{dcterms:description}  \\ \hline
\texttt{Q2} & \texttt{dcterms:publisher}  \\ \hline 
\texttt{Q3} & \texttt{dcterms:issued}  \\ \hline
\texttt{Q4} & \texttt{dcterms:issued}  \\ \hline
\texttt{Q5} & \texttt{dcterms:contributor}  \\ \hline


\end{tabular}
\end{table}



%%%%%%%%%%%%%%%%%%%%%%%%%%%%%
%%%  3.System evaluation  %%%
%%%%%%%%%%%%%%%%%%%%%%%%%%%%%

\section{System Evaluation}
\label{sec:evaluation}



\todo{more section here}\\

%%%%%%%%%%%%%%%%%%%%%%%%%
%%%  4. Related work  %%%
%%%%%%%%%%%%%%%%%%%%%%%%%

%\section{Related Work}\label{sec:soa}


%%%%%%%%%%%%%%%%%%%%%%%%%%%%%%%%%%%%%%%
%%%  5. Conclusion and Future Work  %%%
%%%%%%%%%%%%%%%%%%%%%%%%%%%%%%%%%%%%%%%

\section{Conclusion and Future Work}
\label{sec:conclusion}
%\input{conclusion}

\todo{wrap up}

%%%%%%%%%%%%%%%%%%%%%%%%%
%%%  Acknowledgments  %%%
%%%%%%%%%%%%%%%%%%%%%%%%%
%\vspace{1mm}
\paragraph{\textbf{Acknowledgments}} %\label{sec:acknowledgments}
\todo{thanks the good guys} 


\bibliographystyle{abbrv}
%\nocite{*}
\bibliography{tersan2016}
%\balancecolumns
\end{document}
